% Options for packages loaded elsewhere
\PassOptionsToPackage{unicode}{hyperref}
\PassOptionsToPackage{hyphens}{url}
\PassOptionsToPackage{dvipsnames,svgnames,x11names}{xcolor}
%
\documentclass[
  letterpaper,
  DIV=11,
  numbers=noendperiod]{scrartcl}

\usepackage{amsmath,amssymb}
\usepackage{iftex}
\ifPDFTeX
  \usepackage[T1]{fontenc}
  \usepackage[utf8]{inputenc}
  \usepackage{textcomp} % provide euro and other symbols
\else % if luatex or xetex
  \usepackage{unicode-math}
  \defaultfontfeatures{Scale=MatchLowercase}
  \defaultfontfeatures[\rmfamily]{Ligatures=TeX,Scale=1}
\fi
\usepackage{lmodern}
\ifPDFTeX\else  
    % xetex/luatex font selection
\fi
% Use upquote if available, for straight quotes in verbatim environments
\IfFileExists{upquote.sty}{\usepackage{upquote}}{}
\IfFileExists{microtype.sty}{% use microtype if available
  \usepackage[]{microtype}
  \UseMicrotypeSet[protrusion]{basicmath} % disable protrusion for tt fonts
}{}
\makeatletter
\@ifundefined{KOMAClassName}{% if non-KOMA class
  \IfFileExists{parskip.sty}{%
    \usepackage{parskip}
  }{% else
    \setlength{\parindent}{0pt}
    \setlength{\parskip}{6pt plus 2pt minus 1pt}}
}{% if KOMA class
  \KOMAoptions{parskip=half}}
\makeatother
\usepackage{xcolor}
\setlength{\emergencystretch}{3em} % prevent overfull lines
\setcounter{secnumdepth}{-\maxdimen} % remove section numbering
% Make \paragraph and \subparagraph free-standing
\makeatletter
\ifx\paragraph\undefined\else
  \let\oldparagraph\paragraph
  \renewcommand{\paragraph}{
    \@ifstar
      \xxxParagraphStar
      \xxxParagraphNoStar
  }
  \newcommand{\xxxParagraphStar}[1]{\oldparagraph*{#1}\mbox{}}
  \newcommand{\xxxParagraphNoStar}[1]{\oldparagraph{#1}\mbox{}}
\fi
\ifx\subparagraph\undefined\else
  \let\oldsubparagraph\subparagraph
  \renewcommand{\subparagraph}{
    \@ifstar
      \xxxSubParagraphStar
      \xxxSubParagraphNoStar
  }
  \newcommand{\xxxSubParagraphStar}[1]{\oldsubparagraph*{#1}\mbox{}}
  \newcommand{\xxxSubParagraphNoStar}[1]{\oldsubparagraph{#1}\mbox{}}
\fi
\makeatother


\providecommand{\tightlist}{%
  \setlength{\itemsep}{0pt}\setlength{\parskip}{0pt}}\usepackage{longtable,booktabs,array}
\usepackage{calc} % for calculating minipage widths
% Correct order of tables after \paragraph or \subparagraph
\usepackage{etoolbox}
\makeatletter
\patchcmd\longtable{\par}{\if@noskipsec\mbox{}\fi\par}{}{}
\makeatother
% Allow footnotes in longtable head/foot
\IfFileExists{footnotehyper.sty}{\usepackage{footnotehyper}}{\usepackage{footnote}}
\makesavenoteenv{longtable}
\usepackage{graphicx}
\makeatletter
\newsavebox\pandoc@box
\newcommand*\pandocbounded[1]{% scales image to fit in text height/width
  \sbox\pandoc@box{#1}%
  \Gscale@div\@tempa{\textheight}{\dimexpr\ht\pandoc@box+\dp\pandoc@box\relax}%
  \Gscale@div\@tempb{\linewidth}{\wd\pandoc@box}%
  \ifdim\@tempb\p@<\@tempa\p@\let\@tempa\@tempb\fi% select the smaller of both
  \ifdim\@tempa\p@<\p@\scalebox{\@tempa}{\usebox\pandoc@box}%
  \else\usebox{\pandoc@box}%
  \fi%
}
% Set default figure placement to htbp
\def\fps@figure{htbp}
\makeatother

\usepackage{fvextra}
\DefineVerbatimEnvironment{Highlighting}{Verbatim}{breaklines=true,breakanywhere=true,commandchars=\\\{\}}
\KOMAoption{captions}{tableheading}
\makeatletter
\@ifpackageloaded{caption}{}{\usepackage{caption}}
\AtBeginDocument{%
\ifdefined\contentsname
  \renewcommand*\contentsname{Table of contents}
\else
  \newcommand\contentsname{Table of contents}
\fi
\ifdefined\listfigurename
  \renewcommand*\listfigurename{List of Figures}
\else
  \newcommand\listfigurename{List of Figures}
\fi
\ifdefined\listtablename
  \renewcommand*\listtablename{List of Tables}
\else
  \newcommand\listtablename{List of Tables}
\fi
\ifdefined\figurename
  \renewcommand*\figurename{Figure}
\else
  \newcommand\figurename{Figure}
\fi
\ifdefined\tablename
  \renewcommand*\tablename{Table}
\else
  \newcommand\tablename{Table}
\fi
}
\@ifpackageloaded{float}{}{\usepackage{float}}
\floatstyle{ruled}
\@ifundefined{c@chapter}{\newfloat{codelisting}{h}{lop}}{\newfloat{codelisting}{h}{lop}[chapter]}
\floatname{codelisting}{Listing}
\newcommand*\listoflistings{\listof{codelisting}{List of Listings}}
\makeatother
\makeatletter
\makeatother
\makeatletter
\@ifpackageloaded{caption}{}{\usepackage{caption}}
\@ifpackageloaded{subcaption}{}{\usepackage{subcaption}}
\makeatother

\usepackage{bookmark}

\IfFileExists{xurl.sty}{\usepackage{xurl}}{} % add URL line breaks if available
\urlstyle{same} % disable monospaced font for URLs
\hypersetup{
  colorlinks=true,
  linkcolor={blue},
  filecolor={Maroon},
  citecolor={Blue},
  urlcolor={Blue},
  pdfcreator={LaTeX via pandoc}}


\author{}
\date{}

\begin{document}


Voici une liste de 100 notions clés liées au \textbf{DataOps}, avec des
définitions concises :

\begin{center}\rule{0.5\linewidth}{0.5pt}\end{center}

\subsubsection{\texorpdfstring{\textbf{Concepts
Fondamentaux}}{Concepts Fondamentaux}}\label{concepts-fondamentaux}

\begin{enumerate}
\def\labelenumi{\arabic{enumi}.}
\tightlist
\item
  \textbf{DataOps} : Méthodologie visant à améliorer la collaboration,
  l'automatisation et la qualité des flux de données.\\
\item
  \textbf{Agile Data} : Application des principes Agile (itération,
  feedback) aux projets data.\\
\item
  \textbf{CI/CD (Continuous Integration/Continuous Delivery)} :
  Automatisation du déploiement de pipelines de données.\\
\item
  \textbf{Data Governance} : Gestion des normes, politiques et qualité
  des données.\\
\item
  \textbf{Data Lineage} : Traçabilité des données de la source à la
  destination.\\
\item
  \textbf{Data Catalog} : Inventaire organisé des actifs data d'une
  organisation.\\
\item
  \textbf{Data Observability} : Surveillance proactive de la santé des
  données (fraîcheur, qualité).\\
\item
  \textbf{Data Mesh} : Architecture décentralisée traitant les données
  comme des produits.\\
\item
  \textbf{Data Fabric} : Couche unifiée pour connecter des sources de
  données hétérogènes.\\
\item
  \textbf{Data Pipeline} : Processus automatisé de collecte,
  transformation et livraison des données.
\end{enumerate}

\begin{center}\rule{0.5\linewidth}{0.5pt}\end{center}

\subsubsection{\texorpdfstring{\textbf{Outils \&
Technologies}}{Outils \& Technologies}}\label{outils-technologies}

\begin{enumerate}
\def\labelenumi{\arabic{enumi}.}
\setcounter{enumi}{10}
\tightlist
\item
  \textbf{ETL (Extract, Transform, Load)} : Processus d'intégration de
  données.\\
\item
  \textbf{ELT (Extract, Load, Transform)} : Alternative à l'ETL,
  transformation après chargement.\\
\item
  \textbf{Data Lake} : Stockage de données brutes (structurées ou
  non).\\
\item
  \textbf{Data Warehouse} : Entrepôt de données structurées pour
  l'analyse.\\
\item
  \textbf{Apache Airflow} : Outil d'orchestration de workflows data.\\
\item
  \textbf{dbt (Data Build Tool)} : Framework de transformation de
  données via SQL.\\
\item
  \textbf{Great Expectations} : Bibliothèque de tests et validation des
  données.\\
\item
  \textbf{Kubernetes} : Orchestration de conteneurs pour des pipelines
  scalables.\\
\item
  \textbf{Snowflake} : Plateforme cloud de data warehousing.\\
\item
  \textbf{Databricks} : Solution analytics basée sur Apache Spark.
\end{enumerate}

\begin{center}\rule{0.5\linewidth}{0.5pt}\end{center}

\subsubsection{\texorpdfstring{\textbf{Pratiques
Opérationnelles}}{Pratiques Opérationnelles}}\label{pratiques-opuxe9rationnelles}

\begin{enumerate}
\def\labelenumi{\arabic{enumi}.}
\setcounter{enumi}{20}
\tightlist
\item
  \textbf{Data Versioning} : Gestion des versions des jeux de données et
  du code.\\
\item
  \textbf{Data Testing} : Vérification automatique de la qualité des
  données.\\
\item
  \textbf{Data Monitoring} : Surveillance des performances des
  pipelines.\\
\item
  \textbf{Infrastructure as Code (IaC)} : Gestion de l'infrastructure
  via des fichiers de configuration.\\
\item
  \textbf{Shift-Left Testing} : Tests intégrés tôt dans le cycle de
  développement.\\
\item
  \textbf{Data Replication} : Copie des données entre systèmes pour la
  redondance.\\
\item
  \textbf{Data Virtualization} : Accès unifié à des données sans
  réplication physique.\\
\item
  \textbf{Change Data Capture (CDC)} : Détection et propagation des
  changements de données en temps réel.\\
\item
  \textbf{Data Security} : Protection des données contre les accès non
  autorisés.\\
\item
  \textbf{Data Encryption} : Chiffrement des données au repos ou en
  transit.
\end{enumerate}

\begin{center}\rule{0.5\linewidth}{0.5pt}\end{center}

\subsubsection{\texorpdfstring{\textbf{Qualité \&
Conformité}}{Qualité \& Conformité}}\label{qualituxe9-conformituxe9}

\begin{enumerate}
\def\labelenumi{\arabic{enumi}.}
\setcounter{enumi}{30}
\tightlist
\item
  \textbf{Data Quality} : Mesure de l'exactitude, complétude et
  cohérence des données.\\
\item
  \textbf{Data Cleansing} : Nettoyage des données erronées ou
  incomplètes.\\
\item
  \textbf{Data Validation} : Vérification de la conformité des données
  aux règles métier.\\
\item
  \textbf{Data Profiling} : Analyse statistique des données pour
  identifier des anomalies.\\
\item
  \textbf{GDPR} : Règlement européen sur la protection des données
  personnelles.\\
\item
  \textbf{Data Anonymization} : Suppression des informations
  identifiantes des données.\\
\item
  \textbf{Data Retention Policy} : Politique de conservation/suppression
  des données.\\
\item
  \textbf{Master Data Management (MDM)} : Gestion des données critiques
  de référence (clients, produits).\\
\item
  \textbf{Data Sovereignty} : Conformité aux lois locales sur le
  stockage des données.\\
\item
  \textbf{Data Compliance} : Respect des réglementations sectorielles
  (ex : HIPAA, CCPA).
\end{enumerate}

\begin{center}\rule{0.5\linewidth}{0.5pt}\end{center}

\subsubsection{\texorpdfstring{\textbf{Collaboration \&
Gestion}}{Collaboration \& Gestion}}\label{collaboration-gestion}

\begin{enumerate}
\def\labelenumi{\arabic{enumi}.}
\setcounter{enumi}{40}
\tightlist
\item
  \textbf{Data Democratization} : Accès aux données par des non-experts
  via des outils simples.\\
\item
  \textbf{Data Steward} : Responsable de la qualité et gouvernance des
  données.\\
\item
  \textbf{DataOps Maturity Model} : Évaluation du niveau de maturité
  DataOps d'une organisation.\\
\item
  \textbf{SRE (Site Reliability Engineering)} : Application des
  principes de fiabilité aux systèmes data.\\
\item
  \textbf{Data Literacy} : Compétences des équipes à comprendre et
  utiliser les données.\\
\item
  \textbf{Cross-Functional Teams} : Collaboration entre data engineers,
  scientifiques et métiers.\\
\item
  \textbf{Data Contracts} : Accords formalisés sur le format et la
  livraison des données.\\
\item
  \textbf{Self-Service Analytics} : Outils permettant aux utilisateurs
  de générer des rapports sans assistance.\\
\item
  \textbf{DataOps Platform} : Solution intégrée pour orchestrer les
  workflows data (ex : DataKitchen).\\
\item
  \textbf{Feedback Loop} : Mécanismes d'amélioration continue basés sur
  les retours utilisateurs.
\end{enumerate}

\begin{center}\rule{0.5\linewidth}{0.5pt}\end{center}

\subsubsection{\texorpdfstring{\textbf{Techniques
Avancées}}{Techniques Avancées}}\label{techniques-avancuxe9es}

\begin{enumerate}
\def\labelenumi{\arabic{enumi}.}
\setcounter{enumi}{50}
\tightlist
\item
  \textbf{Data Streaming} : Traitement de données en temps réel (ex :
  Apache Kafka).\\
\item
  \textbf{Batch Processing} : Traitement par lots de gros volumes de
  données.\\
\item
  \textbf{DataOps for ML (MLOps)} : Intégration du DataOps dans les
  cycles de machine learning.\\
\item
  \textbf{Feature Store} : Stockage centralisé de features pour les
  modèles ML.\\
\item
  \textbf{A/B Testing} : Comparaison de versions de modèles ou de
  pipelines.\\
\item
  \textbf{Data Orchestration} : Coordination des tâches entre systèmes
  hétérogènes.\\
\item
  \textbf{Data Schema} : Structure définie des données (ex : JSON
  Schema, Avro).\\
\item
  \textbf{DataOps Dashboard} : Visualisation des métriques clés
  (latence, erreurs).\\
\item
  \textbf{Data Cost Optimization} : Réduction des coûts de stockage et
  traitement.\\
\item
  \textbf{Data Archiving} : Stockage à long terme de données peu
  utilisées.
\end{enumerate}

\begin{center}\rule{0.5\linewidth}{0.5pt}\end{center}

\subsubsection{\texorpdfstring{\textbf{Gestion des Erreurs \&
Résilience}}{Gestion des Erreurs \& Résilience}}\label{gestion-des-erreurs-ruxe9silience}

\begin{enumerate}
\def\labelenumi{\arabic{enumi}.}
\setcounter{enumi}{60}
\tightlist
\item
  \textbf{Error Handling} : Mécanismes de gestion des erreurs dans les
  pipelines.\\
\item
  \textbf{Data Rollback} : Retour à une version antérieure des données
  en cas d'échec.\\
\item
  \textbf{Disaster Recovery} : Plan de restauration des données après un
  incident.\\
\item
  \textbf{Data Backup} : Copies de sauvegarde des données critiques.\\
\item
  \textbf{Chaos Engineering} : Tests proactifs de résilience des
  systèmes data.\\
\item
  \textbf{Retry Mechanisms} : Relance automatique des tâches en échec.\\
\item
  \textbf{Circuit Breaker} : Désactivation temporaire de composants
  défaillants.\\
\item
  \textbf{Data Consistency} : Garantie que les données restent
  cohérentes entre systèmes.\\
\item
  \textbf{Idempotency} : Capacité à relancer une opération sans effets
  secondaires.\\
\item
  \textbf{Data SLAs (Service Level Agreements)} : Engagements sur la
  disponibilité des données.
\end{enumerate}

\begin{center}\rule{0.5\linewidth}{0.5pt}\end{center}

\subsubsection{\texorpdfstring{\textbf{Cloud \&
Scalabilité}}{Cloud \& Scalabilité}}\label{cloud-scalabilituxe9}

\begin{enumerate}
\def\labelenumi{\arabic{enumi}.}
\setcounter{enumi}{70}
\tightlist
\item
  \textbf{Cloud DataOps} : Implémentation du DataOps sur des plateformes
  cloud (AWS, Azure, GCP).\\
\item
  \textbf{Serverless Data Pipelines} : Exécution de pipelines sans
  gestion d'infrastructure.\\
\item
  \textbf{Auto-Scaling} : Ajustement automatique des ressources selon la
  charge.\\
\item
  \textbf{Multi-Cloud Strategy} : Utilisation de plusieurs fournisseurs
  cloud pour éviter le lock-in.\\
\item
  \textbf{DataOps as a Service} : Solutions DataOps managées par un
  tiers.\\
\item
  \textbf{Data Lakehouse} : Combinaison des avantages du Data Lake et du
  Data Warehouse.\\
\item
  \textbf{Data Partitioning} : Division des données pour optimiser les
  performances.\\
\item
  \textbf{Cold Storage} : Stockage low-cost pour données rarement
  consultées.\\
\item
  \textbf{DataOps on Edge} : Traitement des données près de la source
  (IoT).\\
\item
  \textbf{Hybrid DataOps} : Combinaison de systèmes cloud et on-premise.
\end{enumerate}

\begin{center}\rule{0.5\linewidth}{0.5pt}\end{center}

\subsubsection{\texorpdfstring{\textbf{Futures
Tendances}}{Futures Tendances}}\label{futures-tendances}

\begin{enumerate}
\def\labelenumi{\arabic{enumi}.}
\setcounter{enumi}{80}
\tightlist
\item
  \textbf{DataOps Automation} : Utilisation de l'IA pour automatiser des
  tâches DataOps.\\
\item
  \textbf{DataOps for AI} : Intégration des pipelines data dans les
  workflows d'IA générative.\\
\item
  \textbf{Data Collaboration Platforms} : Outils facilitant le travail
  d'équipe sur les données.\\
\item
  \textbf{DataOps Ethics} : Bonnes pratiques pour une utilisation
  responsable des données.\\
\item
  \textbf{Green DataOps} : Réduction de l'empreinte carbone des
  infrastructures data.\\
\item
  \textbf{Data Product Mindset} : Traiter les données comme des produits
  orientés utilisateur.\\
\item
  \textbf{Real-Time DataOps} : Optimisation pour des insights en temps
  réel.\\
\item
  \textbf{DataOps for Unstructured Data} : Gestion de données
  textuelles, images, etc.\\
\item
  \textbf{Augmented Data Management} : Utilisation de l'IA pour
  améliorer la gouvernance.\\
\item
  \textbf{DataOps Sandbox} : Environnements isolés pour tester des
  scénarios data.
\end{enumerate}

\begin{center}\rule{0.5\linewidth}{0.5pt}\end{center}

\subsubsection{\texorpdfstring{\textbf{Bonnes
Pratiques}}{Bonnes Pratiques}}\label{bonnes-pratiques}

\begin{enumerate}
\def\labelenumi{\arabic{enumi}.}
\setcounter{enumi}{90}
\tightlist
\item
  \textbf{Documentation as Code} : Documentation versionnée et intégrée
  aux workflows.\\
\item
  \textbf{DataOps Metrics} : Mesures de performance (ex : temps de
  livraison, taux d'erreur).\\
\item
  \textbf{Root Cause Analysis (RCA)} : Identification systémique des
  causes d'incidents.\\
\item
  \textbf{Incident Management} : Processus structuré pour résoudre les
  problèmes data.\\
\item
  \textbf{DataOps Training} : Formation continue des équipes aux outils
  et méthodes.\\
\item
  \textbf{Cost Monitoring} : Surveillance des coûts liés aux ressources
  data.\\
\item
  \textbf{Data Privacy by Design} : Intégration de la confidentialité
  dès la conception.\\
\item
  \textbf{Data Tokenization} : Remplacement des données sensibles par
  des jetons.\\
\item
  \textbf{DataOps Community} : Partage de connaissances via des forums
  ou meetups.\\
\item
  \textbf{Continuous Improvement} : Optimisation itérative des processus
  DataOps.
\end{enumerate}

\begin{center}\rule{0.5\linewidth}{0.5pt}\end{center}

Cette liste couvre les aspects techniques, organisationnels et
stratégiques du DataOps, reflétant son rôle central dans les
architectures data modernes.

Voici une liste de \textbf{200 notions} couvrant à la fois le
\textbf{DataOps} et \textbf{Kubernetes}, organisées en deux parties
distinctes pour plus de clarté. Les définitions sont concises et
adaptées à un public francophone.

\begin{center}\rule{0.5\linewidth}{0.5pt}\end{center}

\subsubsection{\texorpdfstring{\textbf{Partie 1 : DataOps (100
Notions)}}{Partie 1 : DataOps (100 Notions)}}\label{partie-1-dataops-100-notions}

\paragraph{\texorpdfstring{\textbf{Concepts
Fondamentaux}}{Concepts Fondamentaux}}\label{concepts-fondamentaux-1}

\begin{enumerate}
\def\labelenumi{\arabic{enumi}.}
\tightlist
\item
  \textbf{DataOps} : Méthodologie Agile pour automatiser et optimiser
  les flux de données.\\
\item
  \textbf{Data Pipeline} : Chaîne automatisée de collecte,
  transformation et livraison de données.\\
\item
  \textbf{Data Governance} : Cadre de gestion de la qualité, sécurité et
  conformité des données.\\
\item
  \textbf{Data Lineage} : Traçabilité des données de la source au
  consommateur.\\
\item
  \textbf{Data Mesh} : Architecture décentralisée traitant les données
  comme des produits.\\
\item
  \textbf{Data Fabric} : Couche unifiée pour intégrer des sources
  hétérogènes.\\
\item
  \textbf{Data Catalog} : Inventaire centralisé des métadonnées.\\
\item
  \textbf{Data Observability} : Surveillance proactive de la santé des
  données.\\
\item
  \textbf{Data Quality} : Mesure de l'exactitude, complétude et
  cohérence des données.\\
\item
  \textbf{Data Democratization} : Accès aux données par des non-experts.
\end{enumerate}

\paragraph{\texorpdfstring{\textbf{Outils \&
Technologies}}{Outils \& Technologies}}\label{outils-technologies-1}

\begin{enumerate}
\def\labelenumi{\arabic{enumi}.}
\setcounter{enumi}{10}
\tightlist
\item
  \textbf{ETL (Extract, Transform, Load)} : Intégration de données via
  transformation avant chargement.\\
\item
  \textbf{ELT (Extract, Load, Transform)} : Transformation après
  chargement (ex : cloud).\\
\item
  \textbf{Apache Airflow} : Orchestrateur de workflows data.\\
\item
  \textbf{dbt (Data Build Tool)} : Transformation de données via SQL.\\
\item
  \textbf{Great Expectations} : Validation automatisée des données.\\
\item
  \textbf{Snowflake} : Data Warehouse cloud.\\
\item
  \textbf{Databricks} : Plateforme analytics basée sur Apache Spark.\\
\item
  \textbf{Apache Kafka} : Plateforme de streaming de données en temps
  réel.\\
\item
  \textbf{Talend} : Outil d'intégration et de nettoyage de données.\\
\item
  \textbf{Fivetran} : Solution SaaS pour l'intégration de données.
\end{enumerate}

\paragraph{\texorpdfstring{\textbf{Pratiques \&
Méthodes}}{Pratiques \& Méthodes}}\label{pratiques-muxe9thodes}

\begin{enumerate}
\def\labelenumi{\arabic{enumi}.}
\setcounter{enumi}{20}
\tightlist
\item
  \textbf{CI/CD Data} : Automatisation du déploiement des pipelines.\\
\item
  \textbf{Data Versioning} : Gestion des versions des jeux de données.\\
\item
  \textbf{Data Testing} : Tests automatisés de qualité des données.\\
\item
  \textbf{Shift-Left Testing} : Tests intégrés tôt dans le cycle de
  développement.\\
\item
  \textbf{Data Contracts} : Accords sur le format et la livraison des
  données.\\
\item
  \textbf{Data Replication} : Copie des données pour la redondance.\\
\item
  \textbf{Data Virtualization} : Accès unifié sans réplication
  physique.\\
\item
  \textbf{Data Encryption} : Chiffrement des données au repos/en
  transit.\\
\item
  \textbf{Data Anonymization} : Suppression des identifiants
  personnels.\\
\item
  \textbf{Master Data Management (MDM)} : Gestion des données de
  référence.
\end{enumerate}

\paragraph{\texorpdfstring{\textbf{Qualité \&
Conformité}}{Qualité \& Conformité}}\label{qualituxe9-conformituxe9-1}

\begin{enumerate}
\def\labelenumi{\arabic{enumi}.}
\setcounter{enumi}{30}
\tightlist
\item
  \textbf{GDPR} : Règlement européen sur la protection des données.\\
\item
  \textbf{Data Profiling} : Analyse statistique pour détecter des
  anomalies.\\
\item
  \textbf{Data Cleansing} : Nettoyage des données erronées.\\
\item
  \textbf{Data Retention Policy} : Politique de conservation des
  données.\\
\item
  \textbf{Data Sovereignty} : Conformité aux lois locales de stockage.\\
\item
  \textbf{HIPAA} : Norme de sécurité pour les données de santé.\\
\item
  \textbf{Data Compliance} : Respect des réglementations sectorielles.\\
\item
  \textbf{Data Tokenization} : Remplacement des données sensibles par
  des jetons.\\
\item
  \textbf{Data Privacy by Design} : Confidentialité intégrée dès la
  conception.\\
\item
  \textbf{Data Audit} : Vérification de la conformité et de la sécurité.
\end{enumerate}

\paragraph{\texorpdfstring{\textbf{Collaboration \&
Gestion}}{Collaboration \& Gestion}}\label{collaboration-gestion-1}

\begin{enumerate}
\def\labelenumi{\arabic{enumi}.}
\setcounter{enumi}{40}
\tightlist
\item
  \textbf{Data Steward} : Responsable de la gouvernance des données.\\
\item
  \textbf{Data Literacy} : Compétences en analyse et interprétation des
  données.\\
\item
  \textbf{SRE (Site Reliability Engineering)} : Fiabilité des systèmes
  data.\\
\item
  \textbf{DataOps Maturity Model} : Évaluation du niveau de maturité
  DataOps.\\
\item
  \textbf{Self-Service Analytics} : Outils d'analyse en libre-service.\\
\item
  \textbf{Data Collaboration} : Partage transversal des données.\\
\item
  \textbf{DataOps Platform} : Solution intégrée (ex : DataKitchen).\\
\item
  \textbf{DataOps Metrics} : Mesures de performance (ex : temps de
  livraison).\\
\item
  \textbf{Data Product Mindset} : Approche produit pour les données.\\
\item
  \textbf{Feedback Loop} : Amélioration continue via retours
  utilisateurs.
\end{enumerate}

\paragraph{\texorpdfstring{\textbf{Techniques
Avancées}}{Techniques Avancées}}\label{techniques-avancuxe9es-1}

\begin{enumerate}
\def\labelenumi{\arabic{enumi}.}
\setcounter{enumi}{50}
\tightlist
\item
  \textbf{Data Streaming} : Traitement en temps réel (ex : Kafka).\\
\item
  \textbf{Batch Processing} : Traitement par lots (ex : Hadoop).\\
\item
  \textbf{MLOps} : Intégration du Machine Learning dans les pipelines.\\
\item
  \textbf{Feature Store} : Stockage centralisé de features pour ML.\\
\item
  \textbf{A/B Testing Data} : Comparaison de versions de modèles.\\
\item
  \textbf{Data Schema Evolution} : Gestion des changements de
  structure.\\
\item
  \textbf{Data Archiving} : Stockage long terme de données inactives.\\
\item
  \textbf{Data Cost Optimization} : Réduction des coûts cloud.\\
\item
  \textbf{DataOps for IoT} : Gestion des flux de données IoT.\\
\item
  \textbf{Real-Time Analytics} : Analyse instantanée (ex : Apache
  Flink).
\end{enumerate}

\paragraph{\texorpdfstring{\textbf{Sécurité \&
Résilience}}{Sécurité \& Résilience}}\label{suxe9curituxe9-ruxe9silience}

\begin{enumerate}
\def\labelenumi{\arabic{enumi}.}
\setcounter{enumi}{60}
\tightlist
\item
  \textbf{Data Backup} : Sauvegarde des données critiques.\\
\item
  \textbf{Disaster Recovery} : Plan de restauration post-incident.\\
\item
  \textbf{Data Rollback} : Retour à une version antérieure des
  données.\\
\item
  \textbf{Chaos Engineering} : Tests de résilience des systèmes.\\
\item
  \textbf{Data SLAs} : Engagements sur la disponibilité des données.\\
\item
  \textbf{Zero Trust Data} : Sécurité basée sur une vérification
  permanente.\\
\item
  \textbf{Data Masking} : Masquage des données sensibles.\\
\item
  \textbf{Incident Management} : Gestion structurée des incidents.\\
\item
  \textbf{Root Cause Analysis (RCA)} : Identification des causes
  d'erreurs.\\
\item
  \textbf{Data Integrity} : Garantie de l'exactitude des données.
\end{enumerate}

\paragraph{\texorpdfstring{\textbf{Cloud \&
Scalabilité}}{Cloud \& Scalabilité}}\label{cloud-scalabilituxe9-1}

\begin{enumerate}
\def\labelenumi{\arabic{enumi}.}
\setcounter{enumi}{70}
\tightlist
\item
  \textbf{Cloud DataOps} : DataOps sur AWS, Azure, GCP.\\
\item
  \textbf{Serverless Pipelines} : Pipelines sans gestion de serveur.\\
\item
  \textbf{Data Lakehouse} : Fusion Data Lake + Data Warehouse.\\
\item
  \textbf{Multi-Cloud Strategy} : Utilisation de plusieurs clouds.\\
\item
  \textbf{Auto-Scaling} : Ajustement automatique des ressources.\\
\item
  \textbf{Cold Storage} : Stockage low-cost pour données rarement
  utilisées.\\
\item
  \textbf{Hybrid DataOps} : Combinaison cloud et on-premise.\\
\item
  \textbf{Data Partitioning} : Division des données pour optimiser les
  requêtes.\\
\item
  \textbf{DataOps on Edge} : Traitement près de la source (IoT).\\
\item
  \textbf{Green DataOps} : Réduction de l'empreinte carbone.
\end{enumerate}

\paragraph{\texorpdfstring{\textbf{Futures
Tendances}}{Futures Tendances}}\label{futures-tendances-1}

\begin{enumerate}
\def\labelenumi{\arabic{enumi}.}
\setcounter{enumi}{80}
\tightlist
\item
  \textbf{AI-Driven DataOps} : Automatisation par IA.\\
\item
  \textbf{Data Collaboration Platforms} : Outils de travail
  collaboratif.\\
\item
  \textbf{DataOps Ethics} : Éthique dans l'utilisation des données.\\
\item
  \textbf{Augmented Data Management} : IA pour la gouvernance.\\
\item
  \textbf{DataOps Sandbox} : Environnements de test isolés.\\
\item
  \textbf{DataOps for Unstructured Data} : Gestion de texte, images,
  etc.\\
\item
  \textbf{Data Fabric 2.0} : Intégration de l'IA et du ML.\\
\item
  \textbf{Data Monetization} : Commercialisation des données.\\
\item
  \textbf{DataOps as a Service} : Solutions DataOps managées.\\
\item
  \textbf{Quantum DataOps} : Applications des calculateurs quantiques.
\end{enumerate}

\paragraph{\texorpdfstring{\textbf{Bonnes
Pratiques}}{Bonnes Pratiques}}\label{bonnes-pratiques-1}

\begin{enumerate}
\def\labelenumi{\arabic{enumi}.}
\setcounter{enumi}{90}
\tightlist
\item
  \textbf{Documentation as Code} : Docs versionnées avec le code.\\
\item
  \textbf{DataOps Training} : Formation continue des équipes.\\
\item
  \textbf{Cost Monitoring} : Surveillance des coûts cloud.\\
\item
  \textbf{Data Minimalism} : Collecte uniquement des données
  nécessaires.\\
\item
  \textbf{DataOps Manifesto} : Principes fondateurs du DataOps.\\
\item
  \textbf{Data Discovery} : Outils pour explorer les jeux de données.\\
\item
  \textbf{DataOps Sprint} : Cycles de développement Agile.\\
\item
  \textbf{DataOps Community} : Partage de connaissances via des
  forums.\\
\item
  \textbf{Open Data Initiatives} : Partage public de données.\\
\item
  \textbf{Continuous Improvement} : Optimisation itérative des
  processus.
\end{enumerate}

\begin{center}\rule{0.5\linewidth}{0.5pt}\end{center}

\subsubsection{\texorpdfstring{\textbf{Partie 2 : Kubernetes (100
Notions)}}{Partie 2 : Kubernetes (100 Notions)}}\label{partie-2-kubernetes-100-notions}

\paragraph{\texorpdfstring{\textbf{Concepts de
Base}}{Concepts de Base}}\label{concepts-de-base}

\begin{enumerate}
\def\labelenumi{\arabic{enumi}.}
\tightlist
\item
  \textbf{Kubernetes} : Orchestrateur de conteneurs open source.\\
\item
  \textbf{Cluster} : Ensemble de nœuds exécutant des conteneurs.\\
\item
  \textbf{Node} : Machine physique ou virtuelle dans un cluster.\\
\item
  \textbf{Pod} : Plus petite unité déployable (1+ conteneurs).\\
\item
  \textbf{Deployment} : Gestion des mises à jour des pods.\\
\item
  \textbf{Service} : Point d'accès réseau à un groupe de pods.\\
\item
  \textbf{Namespace} : Partitionnement logique d'un cluster.\\
\item
  \textbf{ConfigMap} : Stockage de configurations non sensibles.\\
\item
  \textbf{Secret} : Stockage de données sensibles (mots de passe,
  clés).\\
\item
  \textbf{Persistent Volume (PV)} : Stockage persistant pour les pods.
\end{enumerate}

\paragraph{\texorpdfstring{\textbf{Architecture}}{Architecture}}\label{architecture}

\begin{enumerate}
\def\labelenumi{\arabic{enumi}.}
\setcounter{enumi}{10}
\tightlist
\item
  \textbf{Control Plane} : Cerveau du cluster (API, scheduler, etc.).\\
\item
  \textbf{kube-apiserver} : Point d'entrée pour les commandes.\\
\item
  \textbf{etcd} : Base de données clé-valeur du cluster.\\
\item
  \textbf{kube-scheduler} : Affectation des pods aux nœuds.\\
\item
  \textbf{kube-controller-manager} : Gestion des boucles de contrôle.\\
\item
  \textbf{kubelet} : Agent exécutant les pods sur les nœuds.\\
\item
  \textbf{kube-proxy} : Gestion du réseau entre les pods.\\
\item
  \textbf{Container Runtime} : Logiciel exécutant les conteneurs (ex :
  Docker).\\
\item
  \textbf{Worker Node} : Nœud exécutant les charges de travail.\\
\item
  \textbf{Master Node} : Nœud hébergeant le control plane.
\end{enumerate}

\paragraph{\texorpdfstring{\textbf{Ressources \&
Gestion}}{Ressources \& Gestion}}\label{ressources-gestion}

\begin{enumerate}
\def\labelenumi{\arabic{enumi}.}
\setcounter{enumi}{20}
\tightlist
\item
  \textbf{ReplicaSet} : Garantit le nombre de pods en cours
  d'exécution.\\
\item
  \textbf{StatefulSet} : Gestion des pods avec état persistant.\\
\item
  \textbf{DaemonSet} : Déploiement d'un pod sur chaque nœud.\\
\item
  \textbf{Job} : Exécution ponctuelle d'une tâche.\\
\item
  \textbf{CronJob} : Tâches planifiées (ex : backups).\\
\item
  \textbf{Horizontal Pod Autoscaler (HPA)} : Ajuste le nombre de pods
  selon la charge.\\
\item
  \textbf{Vertical Pod Autoscaler (VPA)} : Ajuste les ressources des
  pods.\\
\item
  \textbf{Resource Quotas} : Limites de ressources par namespace.\\
\item
  \textbf{LimitRange} : Contraintes sur les ressources des pods.\\
\item
  \textbf{Custom Resource Definition (CRD)} : Définition de ressources
  personnalisées.
\end{enumerate}

\paragraph{\texorpdfstring{\textbf{Réseau \&
Communication}}{Réseau \& Communication}}\label{ruxe9seau-communication}

\begin{enumerate}
\def\labelenumi{\arabic{enumi}.}
\setcounter{enumi}{30}
\tightlist
\item
  \textbf{ClusterIP} : Service interne au cluster.\\
\item
  \textbf{NodePort} : Exposition d'un port sur les nœuds.\\
\item
  \textbf{LoadBalancer} : Service avec équilibrage de charge externe.\\
\item
  \textbf{Ingress} : Gestion du trafic HTTP/HTTPS entrant.\\
\item
  \textbf{Ingress Controller} : Implémentation des règles Ingress (ex :
  Nginx).\\
\item
  \textbf{Network Policy} : Règles de sécurité réseau entre pods.\\
\item
  \textbf{CNI (Container Network Interface)} : Plugins réseau pour
  Kubernetes.\\
\item
  \textbf{Service Mesh} : Gestion avancée du trafic (ex : Istio).\\
\item
  \textbf{DNS Kubernetes} : Résolution de noms entre services.\\
\item
  \textbf{kube-dns/coreDNS} : Service DNS du cluster.
\end{enumerate}

\paragraph{\texorpdfstring{\textbf{Stockage}}{Stockage}}\label{stockage}

\begin{enumerate}
\def\labelenumi{\arabic{enumi}.}
\setcounter{enumi}{40}
\tightlist
\item
  \textbf{Persistent Volume Claim (PVC)} : Demande de stockage
  persistant.\\
\item
  \textbf{StorageClass} : Provisionnement dynamique de volumes.\\
\item
  \textbf{Volume} : Stockage monté dans un pod.\\
\item
  \textbf{CSI (Container Storage Interface)} : Standard pour les plugins
  de stockage.\\
\item
  \textbf{Ephemeral Storage} : Stockage temporaire lié au pod.\\
\item
  \textbf{Local Volume} : Stockage attaché à un nœud spécifique.\\
\item
  \textbf{RWO (ReadWriteOnce)} : Mode d'accès au volume (1 nœud en
  écriture).\\
\item
  \textbf{ROX (ReadOnlyMany)} : Volume accessible en lecture par
  plusieurs nœuds.\\
\item
  \textbf{RWX (ReadWriteMany)} : Volume accessible en écriture par
  plusieurs nœuds.\\
\item
  \textbf{Volume Snapshot} : Sauvegarde d'un volume à un instant T.
\end{enumerate}

\paragraph{\texorpdfstring{\textbf{Sécurité}}{Sécurité}}\label{suxe9curituxe9}

\begin{enumerate}
\def\labelenumi{\arabic{enumi}.}
\setcounter{enumi}{50}
\tightlist
\item
  \textbf{RBAC (Role-Based Access Control)} : Gestion des accès via
  rôles.\\
\item
  \textbf{ServiceAccount} : Compte utilisé par les pods pour
  l'authentification.\\
\item
  \textbf{Pod Security Policy (PSP)} : Règles de sécurité pour les pods
  (déprécié).\\
\item
  \textbf{Pod Security Admission (PSA)} : Remplacement des PSP.\\
\item
  \textbf{NetworkPolicy} : Contrôle du trafic entre pods.\\
\item
  \textbf{kube-bench} : Outil de vérification de la sécurité CIS.\\
\item
  \textbf{Seccomp} : Restriction des appels système des conteneurs.\\
\item
  \textbf{AppArmor} : Profils de sécurité pour les conteneurs.\\
\item
  \textbf{Audit Logging} : Journalisation des activités du cluster.\\
\item
  \textbf{TLS Certificates} : Certificats pour sécuriser les
  communications.
\end{enumerate}

\paragraph{\texorpdfstring{\textbf{Monitoring \&
Logging}}{Monitoring \& Logging}}\label{monitoring-logging}

\begin{enumerate}
\def\labelenumi{\arabic{enumi}.}
\setcounter{enumi}{60}
\tightlist
\item
  \textbf{Prometheus} : Outil de monitoring et d'alerting.\\
\item
  \textbf{Grafana} : Visualisation des métriques (ex : dashboards).\\
\item
  \textbf{EFK Stack} : Elasticsearch, Fluentd, Kibana pour les logs.\\
\item
  \textbf{Loki} : Solution de logging légère par Grafana.\\
\item
  \textbf{kube-state-metrics} : Expose l'état des ressources
  Kubernetes.\\
\item
  \textbf{cAdvisor} : Surveillance des ressources des conteneurs.\\
\item
  \textbf{Helm} : Gestionnaire de packages Kubernetes.\\
\item
  \textbf{Alertmanager} : Gestion des alertes Prometheus.\\
\item
  \textbf{Node Exporter} : Collecte des métriques des nœuds.\\
\item
  \textbf{Jaeger} : Outil de tracing distribué.
\end{enumerate}

\paragraph{\texorpdfstring{\textbf{Outils \&
Écosystème}}{Outils \& Écosystème}}\label{outils-uxe9cosystuxe8me}

\begin{enumerate}
\def\labelenumi{\arabic{enumi}.}
\setcounter{enumi}{70}
\tightlist
\item
  \textbf{Helm Chart} : Modèle de déploiement Kubernetes.\\
\item
  \textbf{Kustomize} : Personnalisation des manifests YAML.\\
\item
  \textbf{Operator Framework} : Création d'opérateurs pour
  applications.\\
\item
  \textbf{Argo CD} : Outil de GitOps pour le déploiement continu.\\
\item
  \textbf{Tekton} : Pipeline CI/CD natif pour Kubernetes.\\
\item
  \textbf{Rancher} : Plateforme de gestion de clusters Kubernetes.\\
\item
  \textbf{Kubectl} : CLI pour interagir avec Kubernetes.\\
\item
  \textbf{Minikube} : Outil pour exécuter un cluster local.\\
\item
  \textbf{Kind (Kubernetes in Docker)} : Cluster Kubernetes dans
  Docker.\\
\item
  \textbf{K3s} : Distribution Kubernetes légère pour IoT/Edge.
\end{enumerate}

\paragraph{\texorpdfstring{\textbf{Bonnes
Pratiques}}{Bonnes Pratiques}}\label{bonnes-pratiques-2}

\begin{enumerate}
\def\labelenumi{\arabic{enumi}.}
\setcounter{enumi}{80}
\tightlist
\item
  \textbf{Immutable Containers} : Pas de modifications en runtime.\\
\item
  \textbf{Health Checks} : Liveness et readiness probes.\\
\item
  \textbf{Resource Limits} : Définition des limites CPU/RAM.\\
\item
  \textbf{Pod Anti-Affinity} : Répartition des pods sur différents
  nœuds.\\
\item
  \textbf{Graceful Shutdown} : Arrêt propre des pods.\\
\item
  \textbf{Backup \& Restore} : Sauvegarde de l'état du cluster (ex :
  Velero).\\
\item
  \textbf{GitOps} : Gestion des déploiements via Git.\\
\item
  \textbf{Infrastructure as Code (IaC)} : Déclaration des ressources en
  YAML.\\
\item
  \textbf{Chaos Engineering} : Tests de résilience (ex : Chaos Mesh).\\
\item
  \textbf{Cost Optimization} : Réduction des coûts cloud (ex :
  autoscaling).
\end{enumerate}

\paragraph{\texorpdfstring{\textbf{Cas d'Usage
DataOps}}{Cas d'Usage DataOps}}\label{cas-dusage-dataops}

\begin{enumerate}
\def\labelenumi{\arabic{enumi}.}
\setcounter{enumi}{90}
\tightlist
\item
  \textbf{Spark on Kubernetes} : Exécution d'Apache Spark via des
  pods.\\
\item
  \textbf{Airflow on Kubernetes} : Orchestration de workflows avec
  KubernetesExecutor.\\
\item
  \textbf{Kafka in Kubernetes} : Déploiement de clusters Kafka.\\
\item
  \textbf{MLOps avec Kubeflow} : Plateforme de ML sur Kubernetes.\\
\item
  \textbf{Data Pipelines as Jobs} : Exécution de tâches data via des
  CronJobs.\\
\item
  \textbf{Database Operators} : Gestion de bases de données (ex :
  PostgreSQL Operator).\\
\item
  \textbf{Streaming avec Flink} : Déploiement d'Apache Flink sur
  Kubernetes.\\
\item
  \textbf{Monitoring des Pipelines} : Intégration Prometheus/Grafana.\\
\item
  \textbf{Serverless Data Processing} : Knative pour des fonctions
  serverless.\\
\item
  \textbf{Hybrid Cloud DataOps} : Cluster multi-cloud pour la data.
\end{enumerate}

\begin{center}\rule{0.5\linewidth}{0.5pt}\end{center}

Cette liste combine les concepts clés du \textbf{DataOps} (optimisation
des flux de données) et de \textbf{Kubernetes} (orchestration de
conteneurs), essentiels pour les architectures cloud modernes. Les deux
domaines se complètent pour créer des systèmes data




\end{document}
