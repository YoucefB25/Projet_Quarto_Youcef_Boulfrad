% Options for packages loaded elsewhere
\PassOptionsToPackage{unicode}{hyperref}
\PassOptionsToPackage{hyphens}{url}
\PassOptionsToPackage{dvipsnames,svgnames,x11names}{xcolor}
%
\documentclass[
]{article}

\usepackage{amsmath,amssymb}
\usepackage{iftex}
\ifPDFTeX
  \usepackage[T1]{fontenc}
  \usepackage[utf8]{inputenc}
  \usepackage{textcomp} % provide euro and other symbols
\else % if luatex or xetex
  \usepackage{unicode-math}
  \defaultfontfeatures{Scale=MatchLowercase}
  \defaultfontfeatures[\rmfamily]{Ligatures=TeX,Scale=1}
\fi
\usepackage{lmodern}
\ifPDFTeX\else  
    % xetex/luatex font selection
\fi
% Use upquote if available, for straight quotes in verbatim environments
\IfFileExists{upquote.sty}{\usepackage{upquote}}{}
\IfFileExists{microtype.sty}{% use microtype if available
  \usepackage[]{microtype}
  \UseMicrotypeSet[protrusion]{basicmath} % disable protrusion for tt fonts
}{}
\makeatletter
\@ifundefined{KOMAClassName}{% if non-KOMA class
  \IfFileExists{parskip.sty}{%
    \usepackage{parskip}
  }{% else
    \setlength{\parindent}{0pt}
    \setlength{\parskip}{6pt plus 2pt minus 1pt}}
}{% if KOMA class
  \KOMAoptions{parskip=half}}
\makeatother
\usepackage{xcolor}
\setlength{\emergencystretch}{3em} % prevent overfull lines
\setcounter{secnumdepth}{-\maxdimen} % remove section numbering
% Make \paragraph and \subparagraph free-standing
\makeatletter
\ifx\paragraph\undefined\else
  \let\oldparagraph\paragraph
  \renewcommand{\paragraph}{
    \@ifstar
      \xxxParagraphStar
      \xxxParagraphNoStar
  }
  \newcommand{\xxxParagraphStar}[1]{\oldparagraph*{#1}\mbox{}}
  \newcommand{\xxxParagraphNoStar}[1]{\oldparagraph{#1}\mbox{}}
\fi
\ifx\subparagraph\undefined\else
  \let\oldsubparagraph\subparagraph
  \renewcommand{\subparagraph}{
    \@ifstar
      \xxxSubParagraphStar
      \xxxSubParagraphNoStar
  }
  \newcommand{\xxxSubParagraphStar}[1]{\oldsubparagraph*{#1}\mbox{}}
  \newcommand{\xxxSubParagraphNoStar}[1]{\oldsubparagraph{#1}\mbox{}}
\fi
\makeatother


\providecommand{\tightlist}{%
  \setlength{\itemsep}{0pt}\setlength{\parskip}{0pt}}\usepackage{longtable,booktabs,array}
\usepackage{calc} % for calculating minipage widths
% Correct order of tables after \paragraph or \subparagraph
\usepackage{etoolbox}
\makeatletter
\patchcmd\longtable{\par}{\if@noskipsec\mbox{}\fi\par}{}{}
\makeatother
% Allow footnotes in longtable head/foot
\IfFileExists{footnotehyper.sty}{\usepackage{footnotehyper}}{\usepackage{footnote}}
\makesavenoteenv{longtable}
\usepackage{graphicx}
\makeatletter
\newsavebox\pandoc@box
\newcommand*\pandocbounded[1]{% scales image to fit in text height/width
  \sbox\pandoc@box{#1}%
  \Gscale@div\@tempa{\textheight}{\dimexpr\ht\pandoc@box+\dp\pandoc@box\relax}%
  \Gscale@div\@tempb{\linewidth}{\wd\pandoc@box}%
  \ifdim\@tempb\p@<\@tempa\p@\let\@tempa\@tempb\fi% select the smaller of both
  \ifdim\@tempa\p@<\p@\scalebox{\@tempa}{\usebox\pandoc@box}%
  \else\usebox{\pandoc@box}%
  \fi%
}
% Set default figure placement to htbp
\def\fps@figure{htbp}
\makeatother

\usepackage{fvextra}
\DefineVerbatimEnvironment{Highlighting}{Verbatim}{breaklines=true,breakanywhere=true,commandchars=\\\{\}}
\usepackage[a4paper,margin=1.5cm]{geometry} % Réduction des marges
\usepackage{graphicx}
\usepackage{array}
\renewcommand{\arraystretch}{1.2} % Hauteur des lignes ajustée
\setlength{\tabcolsep}{8pt} % Espacement entre les colonnes
\usepackage{adjustbox} % Ajustement automatique des tableaux
\usepackage{fontenc}
\usepackage{lmodern}
\usepackage{caption}
\captionsetup{font=small} % Taille des légendes réduite
\usepackage{helvet} % Police plus compacte
\renewcommand{\familydefault}{\sfdefault} % Application de la police compacte
\fontsize{9pt}{11pt}\selectfont % Police réduite
\makeatletter
\@ifpackageloaded{caption}{}{\usepackage{caption}}
\AtBeginDocument{%
\ifdefined\contentsname
  \renewcommand*\contentsname{Table of contents}
\else
  \newcommand\contentsname{Table of contents}
\fi
\ifdefined\listfigurename
  \renewcommand*\listfigurename{List of Figures}
\else
  \newcommand\listfigurename{List of Figures}
\fi
\ifdefined\listtablename
  \renewcommand*\listtablename{List of Tables}
\else
  \newcommand\listtablename{List of Tables}
\fi
\ifdefined\figurename
  \renewcommand*\figurename{Figure}
\else
  \newcommand\figurename{Figure}
\fi
\ifdefined\tablename
  \renewcommand*\tablename{Table}
\else
  \newcommand\tablename{Table}
\fi
}
\@ifpackageloaded{float}{}{\usepackage{float}}
\floatstyle{ruled}
\@ifundefined{c@chapter}{\newfloat{codelisting}{h}{lop}}{\newfloat{codelisting}{h}{lop}[chapter]}
\floatname{codelisting}{Listing}
\newcommand*\listoflistings{\listof{codelisting}{List of Listings}}
\makeatother
\makeatletter
\makeatother
\makeatletter
\@ifpackageloaded{caption}{}{\usepackage{caption}}
\@ifpackageloaded{subcaption}{}{\usepackage{subcaption}}
\makeatother

\usepackage{bookmark}

\IfFileExists{xurl.sty}{\usepackage{xurl}}{} % add URL line breaks if available
\urlstyle{same} % disable monospaced font for URLs
\hypersetup{
  pdftitle={Comparaison des Méthodes de Classification},
  colorlinks=true,
  linkcolor={blue},
  filecolor={Maroon},
  citecolor={Blue},
  urlcolor={Blue},
  pdfcreator={LaTeX via pandoc}}


\title{Comparaison des Méthodes de Classification}
\author{}
\date{}

\begin{document}
\maketitle


\subsection{Comparaison des Performances des
Modèles}\label{comparaison-des-performances-des-moduxe8les}

\begin{adjustbox}{max width=\textwidth}
\begin{tabular}{|c|c|c|c|c|c|c|c|c|c|c|c|c|}
\hline
Classe (effectif) & KNN & LDA & QDA & Bayesien Naïf & Arbre CART & Forêt Aléatoire & Reg Log OVA & Reg Log OVO & Reg Multinom & Réseau Neurones & SVM OVA & SVM OVO \\
\hline
1 (10592) & 78.48 & 63.43 & 54.88 & 61.35 & 74.61 & 82.73 & 63.10 & 66.97 & 66.82 & 78.29 & 68.29 & 70.55 \\
2 (14165) & 80.76 & 65.83 & 55.10 & 61.14 & 76.84 & 87.89 & 75.89 & 75.40 & 75.61 & 86.69 & 78.57 & 78.61 \\
3 (7151) & 86.36 & 63.08 & 66.29 & 65.66 & 85.03 & 94.55 & 88.81 & 86.78 & 87.34 & 84.55 & 90.63 & 90.35 \\
4 (549) & 63.64 & 48.18 & 48.18 & 60.00 & 67.27 & 64.55 & 26.36 & 34.55 & 30.91 & 70.00 & 20.00 & 21.82 \\
5 (1899) & 77.89 & 47.11 & 50.26 & 46.32 & 69.21 & 71.84 & 16.58 & 25.79 & 24.21 & 83.95 & 33.16 & 34.47 \\
6 (3473) & 73.05 & 52.45 & 48.27 & 44.81 & 74.06 & 78.67 & 27.67 & 40.06 & 35.30 & 86.31 & 38.33 & 38.62 \\
7 (4102) & 93.79 & 80.88 & 80.88 & 79.29 & 90.86 & 94.28 & 81.36 & 80.39 & 79.66 & 92.33 & 82.10 & 81.49 \\
Total (41931) & 81.42 & 64.04 & 58.60 & 61.70 & 78.35 & 86.55 & 68.07 & 69.99 & 69.54 & 84.38 & 72.22 & 72.80 \\
\hline
\end{tabular}
\end{adjustbox}

\begin{center}\rule{0.5\linewidth}{0.5pt}\end{center}

\section{Meilleures Performances
Globales}\label{meilleures-performances-globales}

\begin{itemize}
\tightlist
\item
  La \textbf{Forêt Aléatoire} est le meilleur modèle global avec
  \textbf{86.55\%} de précision moyenne.
\item
  Le \textbf{Réseau de Neurones} suit de près avec \textbf{84.38\%},
  prouvant l'efficacité des méthodes d'apprentissage profond.
\item
  Le \textbf{KNN (81.42\%)} et l'\textbf{Arbre CART (78.35\%)} sont
  également compétitifs.
\end{itemize}

\section{Analyse des Classes Minoritaires (4, 5,
6)}\label{analyse-des-classes-minoritaires-4-5-6}

\begin{itemize}
\tightlist
\item
  Les classes peu représentées sont souvent mal classées.
\item
  \textbf{Forêt Aléatoire} (64.55\%, 71.84\%, 78.67\%) offre la
  meilleure robustesse.
\item
  \textbf{Réseau de Neurones} (70.00\%, 83.95\%, 86.31\%) s'adapte bien
  aux déséquilibres.
\item
  \textbf{Arbre CART} (67.27\%, 69.21\%, 74.06\%) est un compromis
  intéressant.
\end{itemize}

\section{Méthodes les Plus Faibles
:}\label{muxe9thodes-les-plus-faibles}

\begin{itemize}
\tightlist
\item
  \textbf{SVM et Régressions Logistiques} (OVA et OVO) sous-performent
  sur les classes minoritaires (\textbf{moins de 40\% pour certaines}).
\item
  \textbf{QDA et Bayésien Naïf} sont globalement moins efficaces
  (\textbf{58.60\% et 61.70\%} respectivement).
\end{itemize}

\section{Méthodes Paramétriques vs
Non-Paramétriques}\label{muxe9thodes-paramuxe9triques-vs-non-paramuxe9triques}

Les modèles \textbf{non-paramétriques surpassent largement les modèles
paramétriques} : - \textbf{Forêt Aléatoire (86.55\%)}, \textbf{Réseau de
Neurones (84.38\%)}, \textbf{KNN (81.42\%)} et \textbf{Arbre CART
(78.35\%)} dominent le classement. - À l'inverse, \textbf{LDA
(64.04\%)}, \textbf{QDA (58.60\%)} et \textbf{Bayésien Naïf (61.70\%)}
sont nettement moins performants.

Les \textbf{modèles non-paramétriques} sont plus flexibles et capturent
mieux des structures complexes dans les données, tandis que les modèles
paramétriques reposent sur des hypothèses restrictives.

\section{Multiclasse Natif vs Adapté
(OVA/OVO)}\label{multiclasse-natif-vs-adaptuxe9-ovaovo}

Les méthodes \textbf{multiclasse natives} (comme \textbf{Forêt
Aléatoire, Arbre CART, Réseau de Neurones}) ont des performances
meilleures que les modèles binaires adaptés \textbf{OVA et OVO}.

\begin{itemize}
\tightlist
\item
  \textbf{Forêt Aléatoire (86.55\%)} et \textbf{Réseau de Neurones
  (84.38\%)}, qui sont naturellement adaptés au multiclasse, surpassent
  les modèles \textbf{SVM OVA (72.22\%)} et \textbf{SVM OVO (72.80\%)},
  ainsi que les \textbf{régressions logistiques OVA et OVO}.
\item
  Les méthodes \textbf{binaires adaptées} (OVA et OVO) peinent surtout
  sur les classes minoritaires, avec des scores très faibles
  (\textbf{ex. SVM OVA : 20.00\% sur la classe 4 !}).
\end{itemize}

\subsection{Explication :}\label{explication}

\begin{itemize}
\tightlist
\item
  \textbf{OVA} force une classe unique à se démarquer contre toutes les
  autres.
\item
  \textbf{OVO} compare les classes deux à deux, ce qui est sous-optimal
  pour des classes déséquilibrées.
\end{itemize}

\section{Résumé Final}\label{ruxe9sumuxe9-final}

\begin{itemize}
\tightlist
\item
  \textbf{Les modèles non-paramétriques sont les meilleurs} grâce à leur
  flexibilité et leur adaptation aux classes déséquilibrées.
\item
  \textbf{Les méthodes multiclasses natives (Forêt Aléatoire, Réseau de
  Neurones, Arbre CART) dominent} les modèles binaires adaptés.
\item
  \textbf{Si les classes minoritaires sont importantes}, privilégiez
  \textbf{Forêt Aléatoire, Réseau de Neurones ou Arbre CART}.
\item
  \textbf{Les modèles OVA/OVO ne sont pas adaptés} aux jeux de données
  avec des classes déséquilibrées.
\end{itemize}

\begin{center}\rule{0.5\linewidth}{0.5pt}\end{center}




\end{document}
