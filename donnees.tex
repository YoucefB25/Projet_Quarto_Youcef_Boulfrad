% Options for packages loaded elsewhere
\PassOptionsToPackage{unicode}{hyperref}
\PassOptionsToPackage{hyphens}{url}
\PassOptionsToPackage{dvipsnames,svgnames,x11names}{xcolor}
%
\documentclass[
  letterpaper,
  DIV=11,
  numbers=noendperiod]{scrartcl}

\usepackage{amsmath,amssymb}
\usepackage{iftex}
\ifPDFTeX
  \usepackage[T1]{fontenc}
  \usepackage[utf8]{inputenc}
  \usepackage{textcomp} % provide euro and other symbols
\else % if luatex or xetex
  \usepackage{unicode-math}
  \defaultfontfeatures{Scale=MatchLowercase}
  \defaultfontfeatures[\rmfamily]{Ligatures=TeX,Scale=1}
\fi
\usepackage{lmodern}
\ifPDFTeX\else  
    % xetex/luatex font selection
\fi
% Use upquote if available, for straight quotes in verbatim environments
\IfFileExists{upquote.sty}{\usepackage{upquote}}{}
\IfFileExists{microtype.sty}{% use microtype if available
  \usepackage[]{microtype}
  \UseMicrotypeSet[protrusion]{basicmath} % disable protrusion for tt fonts
}{}
\makeatletter
\@ifundefined{KOMAClassName}{% if non-KOMA class
  \IfFileExists{parskip.sty}{%
    \usepackage{parskip}
  }{% else
    \setlength{\parindent}{0pt}
    \setlength{\parskip}{6pt plus 2pt minus 1pt}}
}{% if KOMA class
  \KOMAoptions{parskip=half}}
\makeatother
\usepackage{xcolor}
\setlength{\emergencystretch}{3em} % prevent overfull lines
\setcounter{secnumdepth}{-\maxdimen} % remove section numbering
% Make \paragraph and \subparagraph free-standing
\makeatletter
\ifx\paragraph\undefined\else
  \let\oldparagraph\paragraph
  \renewcommand{\paragraph}{
    \@ifstar
      \xxxParagraphStar
      \xxxParagraphNoStar
  }
  \newcommand{\xxxParagraphStar}[1]{\oldparagraph*{#1}\mbox{}}
  \newcommand{\xxxParagraphNoStar}[1]{\oldparagraph{#1}\mbox{}}
\fi
\ifx\subparagraph\undefined\else
  \let\oldsubparagraph\subparagraph
  \renewcommand{\subparagraph}{
    \@ifstar
      \xxxSubParagraphStar
      \xxxSubParagraphNoStar
  }
  \newcommand{\xxxSubParagraphStar}[1]{\oldsubparagraph*{#1}\mbox{}}
  \newcommand{\xxxSubParagraphNoStar}[1]{\oldsubparagraph{#1}\mbox{}}
\fi
\makeatother

\usepackage{color}
\usepackage{fancyvrb}
\newcommand{\VerbBar}{|}
\newcommand{\VERB}{\Verb[commandchars=\\\{\}]}
\DefineVerbatimEnvironment{Highlighting}{Verbatim}{commandchars=\\\{\}}
% Add ',fontsize=\small' for more characters per line
\newenvironment{Shaded}{}{}
\newcommand{\AlertTok}[1]{\textcolor[rgb]{1.00,0.00,0.00}{\textbf{#1}}}
\newcommand{\AnnotationTok}[1]{\textcolor[rgb]{0.38,0.63,0.69}{\textbf{\textit{#1}}}}
\newcommand{\AttributeTok}[1]{\textcolor[rgb]{0.49,0.56,0.16}{#1}}
\newcommand{\BaseNTok}[1]{\textcolor[rgb]{0.25,0.63,0.44}{#1}}
\newcommand{\BuiltInTok}[1]{\textcolor[rgb]{0.00,0.50,0.00}{#1}}
\newcommand{\CharTok}[1]{\textcolor[rgb]{0.25,0.44,0.63}{#1}}
\newcommand{\CommentTok}[1]{\textcolor[rgb]{0.38,0.63,0.69}{\textit{#1}}}
\newcommand{\CommentVarTok}[1]{\textcolor[rgb]{0.38,0.63,0.69}{\textbf{\textit{#1}}}}
\newcommand{\ConstantTok}[1]{\textcolor[rgb]{0.53,0.00,0.00}{#1}}
\newcommand{\ControlFlowTok}[1]{\textcolor[rgb]{0.00,0.44,0.13}{\textbf{#1}}}
\newcommand{\DataTypeTok}[1]{\textcolor[rgb]{0.56,0.13,0.00}{#1}}
\newcommand{\DecValTok}[1]{\textcolor[rgb]{0.25,0.63,0.44}{#1}}
\newcommand{\DocumentationTok}[1]{\textcolor[rgb]{0.73,0.13,0.13}{\textit{#1}}}
\newcommand{\ErrorTok}[1]{\textcolor[rgb]{1.00,0.00,0.00}{\textbf{#1}}}
\newcommand{\ExtensionTok}[1]{#1}
\newcommand{\FloatTok}[1]{\textcolor[rgb]{0.25,0.63,0.44}{#1}}
\newcommand{\FunctionTok}[1]{\textcolor[rgb]{0.02,0.16,0.49}{#1}}
\newcommand{\ImportTok}[1]{\textcolor[rgb]{0.00,0.50,0.00}{\textbf{#1}}}
\newcommand{\InformationTok}[1]{\textcolor[rgb]{0.38,0.63,0.69}{\textbf{\textit{#1}}}}
\newcommand{\KeywordTok}[1]{\textcolor[rgb]{0.00,0.44,0.13}{\textbf{#1}}}
\newcommand{\NormalTok}[1]{#1}
\newcommand{\OperatorTok}[1]{\textcolor[rgb]{0.40,0.40,0.40}{#1}}
\newcommand{\OtherTok}[1]{\textcolor[rgb]{0.00,0.44,0.13}{#1}}
\newcommand{\PreprocessorTok}[1]{\textcolor[rgb]{0.74,0.48,0.00}{#1}}
\newcommand{\RegionMarkerTok}[1]{#1}
\newcommand{\SpecialCharTok}[1]{\textcolor[rgb]{0.25,0.44,0.63}{#1}}
\newcommand{\SpecialStringTok}[1]{\textcolor[rgb]{0.73,0.40,0.53}{#1}}
\newcommand{\StringTok}[1]{\textcolor[rgb]{0.25,0.44,0.63}{#1}}
\newcommand{\VariableTok}[1]{\textcolor[rgb]{0.10,0.09,0.49}{#1}}
\newcommand{\VerbatimStringTok}[1]{\textcolor[rgb]{0.25,0.44,0.63}{#1}}
\newcommand{\WarningTok}[1]{\textcolor[rgb]{0.38,0.63,0.69}{\textbf{\textit{#1}}}}

\providecommand{\tightlist}{%
  \setlength{\itemsep}{0pt}\setlength{\parskip}{0pt}}\usepackage{longtable,booktabs,array}
\usepackage{calc} % for calculating minipage widths
% Correct order of tables after \paragraph or \subparagraph
\usepackage{etoolbox}
\makeatletter
\patchcmd\longtable{\par}{\if@noskipsec\mbox{}\fi\par}{}{}
\makeatother
% Allow footnotes in longtable head/foot
\IfFileExists{footnotehyper.sty}{\usepackage{footnotehyper}}{\usepackage{footnote}}
\makesavenoteenv{longtable}
\usepackage{graphicx}
\makeatletter
\newsavebox\pandoc@box
\newcommand*\pandocbounded[1]{% scales image to fit in text height/width
  \sbox\pandoc@box{#1}%
  \Gscale@div\@tempa{\textheight}{\dimexpr\ht\pandoc@box+\dp\pandoc@box\relax}%
  \Gscale@div\@tempb{\linewidth}{\wd\pandoc@box}%
  \ifdim\@tempb\p@<\@tempa\p@\let\@tempa\@tempb\fi% select the smaller of both
  \ifdim\@tempa\p@<\p@\scalebox{\@tempa}{\usebox\pandoc@box}%
  \else\usebox{\pandoc@box}%
  \fi%
}
% Set default figure placement to htbp
\def\fps@figure{htbp}
\makeatother

\usepackage{fvextra}
\DefineVerbatimEnvironment{Highlighting}{Verbatim}{breaklines=true,breakanywhere=true,commandchars=\\\{\}}
\KOMAoption{captions}{tableheading}
\makeatletter
\@ifpackageloaded{caption}{}{\usepackage{caption}}
\AtBeginDocument{%
\ifdefined\contentsname
  \renewcommand*\contentsname{Table of contents}
\else
  \newcommand\contentsname{Table of contents}
\fi
\ifdefined\listfigurename
  \renewcommand*\listfigurename{List of Figures}
\else
  \newcommand\listfigurename{List of Figures}
\fi
\ifdefined\listtablename
  \renewcommand*\listtablename{List of Tables}
\else
  \newcommand\listtablename{List of Tables}
\fi
\ifdefined\figurename
  \renewcommand*\figurename{Figure}
\else
  \newcommand\figurename{Figure}
\fi
\ifdefined\tablename
  \renewcommand*\tablename{Table}
\else
  \newcommand\tablename{Table}
\fi
}
\@ifpackageloaded{float}{}{\usepackage{float}}
\floatstyle{ruled}
\@ifundefined{c@chapter}{\newfloat{codelisting}{h}{lop}}{\newfloat{codelisting}{h}{lop}[chapter]}
\floatname{codelisting}{Listing}
\newcommand*\listoflistings{\listof{codelisting}{List of Listings}}
\makeatother
\makeatletter
\makeatother
\makeatletter
\@ifpackageloaded{caption}{}{\usepackage{caption}}
\@ifpackageloaded{subcaption}{}{\usepackage{subcaption}}
\makeatother

\usepackage{bookmark}

\IfFileExists{xurl.sty}{\usepackage{xurl}}{} % add URL line breaks if available
\urlstyle{same} % disable monospaced font for URLs
\hypersetup{
  pdftitle={Téléchargement et Préparation du Dataset Covertype},
  colorlinks=true,
  linkcolor={blue},
  filecolor={Maroon},
  citecolor={Blue},
  urlcolor={Blue},
  pdfcreator={LaTeX via pandoc}}


\title{Téléchargement et Préparation du Dataset Covertype}
\author{}
\date{}

\begin{document}
\maketitle


\section{Présentation de la Base de Données
Covertype}\label{pruxe9sentation-de-la-base-de-donnuxe9es-covertype}

La base de données \textbf{Covertype} provient de
l'\href{https://archive.ics.uci.edu/ml/datasets/Covertype}{UCI Machine
Learning Repository}. Elle est utilisée pour \textbf{classer les types
de couvert forestier} à partir de mesures cartographiques (sol,
altitude, pente, distance aux points d'eau, etc.).

\subsection{Caractéristiques du
Dataset}\label{caractuxe9ristiques-du-dataset}

\begin{itemize}
\tightlist
\item
  \textbf{Nombre d'observations} : 581 012
\item
  \textbf{Nombre de variables} : 54 (features continues et binaires)
\item
  \textbf{Nombre de classes} : 7 types de couverture forestière (1 à 7)
\item
  \textbf{Problème à résoudre} : Classification supervisée
\end{itemize}

Les classes ne sont \textbf{pas équilibrées}, ce qui peut influencer la
performance des modèles de classification. Les proportions des classes
dans l'ensemble original sont les suivantes :

\begin{longtable}[]{@{}llll@{}}
\toprule\noalign{}
Classe & Type de forêt & Effectif & Proportion (\%) \\
\midrule\noalign{}
\endhead
\bottomrule\noalign{}
\endlastfoot
1 & Épicéa & 211 840 & 36.5 \\
2 & Pin & 283 301 & 48.8 \\
3 & Peuplier & 35 754 & 6.2 \\
4 & Bouleau & 2 747 & 0.5 \\
5 & Érable & 9 493 & 1.6 \\
6 & Hêtre & 17 367 & 3.0 \\
7 & Mélèze & 18 510 & 3.2 \\
\end{longtable}

\begin{center}\rule{0.5\linewidth}{0.5pt}\end{center}

\subsection{Objectif de
l'Échantillonnage}\label{objectif-de-luxe9chantillonnage}

\subsubsection{Pourquoi réduire la taille du dataset
?}\label{pourquoi-ruxe9duire-la-taille-du-dataset}

Le dataset \textbf{Covertype est très grand} (581 012 individus). En
raison du \textbf{temps de calcul important}, nous avons décidé
d'utiliser un \textbf{échantillon plus petit}, tout en conservant la
distribution des classes.

\subsubsection{Comment gérer le déséquilibre des classes
?}\label{comment-guxe9rer-le-duxe9suxe9quilibre-des-classes}

Certaines classes sont \textbf{très majoritaires} (ex : \textbf{Pin et
Épicéa} représentent à eux seuls \textbf{85\% des données}), tandis que
d'autres sont \textbf{très minoritaires} (ex : \textbf{Bouleau} à
seulement \textbf{0.5\%}).\\
Nous avons appliqué un \textbf{échantillonnage différencié} :

\begin{itemize}
\tightlist
\item
  \textbf{Sous-échantillonnage des classes majoritaires} (Épicéa et Pin)
  → \textbf{5\% de leurs effectifs d'origine}
\item
  \textbf{Sur-échantillonnage relatif des classes minoritaires}
  (Peuplier, Bouleau, Érable, Hêtre, Mélèze) → \textbf{20\% de leurs
  effectifs d'origine}
\end{itemize}

\textbf{Nous ne supprimons pas totalement le déséquilibre}, car nous
souhaitons \textbf{tester nos modèles dans des conditions réalistes}, où
certaines classes restent plus rares que d'autres.

L'échantillon obtenu comptera un peu plus de 40 000 individus, dont 60\%
seront réservés à l'entrainement des modèles, 20\% à la validation des
hyperparamètres, et 20\% aux tests.

\begin{center}\rule{0.5\linewidth}{0.5pt}\end{center}

\subsection{Téléchargement et Préparation des
Données}\label{tuxe9luxe9chargement-et-pruxe9paration-des-donnuxe9es}

\begin{Shaded}
\begin{Highlighting}[]
\ImportTok{import}\NormalTok{ pandas }\ImportTok{as}\NormalTok{ pd}
\ImportTok{from}\NormalTok{ sklearn.model\_selection }\ImportTok{import}\NormalTok{ train\_test\_split}

\CommentTok{\# Téléchargement direct des données depuis l\textquotesingle{}URL}
\NormalTok{url }\OperatorTok{=} \StringTok{"https://archive.ics.uci.edu/ml/machine{-}learning{-}databases/covtype/covtype.data.gz"}
\NormalTok{column\_names }\OperatorTok{=}\NormalTok{ [}\SpecialStringTok{f\textquotesingle{}Feature\_}\SpecialCharTok{\{}\NormalTok{i}\SpecialCharTok{\}}\SpecialStringTok{\textquotesingle{}} \ControlFlowTok{for}\NormalTok{ i }\KeywordTok{in} \BuiltInTok{range}\NormalTok{(}\DecValTok{1}\NormalTok{, }\DecValTok{55}\NormalTok{)] }\OperatorTok{+}\NormalTok{ [}\StringTok{\textquotesingle{}Cover\_Type\textquotesingle{}}\NormalTok{]}
\NormalTok{data }\OperatorTok{=}\NormalTok{ pd.read\_csv(url, header}\OperatorTok{=}\VariableTok{None}\NormalTok{, names}\OperatorTok{=}\NormalTok{column\_names)}

\CommentTok{\# Définition des taux d\textquotesingle{}échantillonnage (différent selon les classes)}
\NormalTok{sampling\_rates }\OperatorTok{=}\NormalTok{ \{}\DecValTok{1}\NormalTok{: }\FloatTok{0.05}\NormalTok{, }\DecValTok{2}\NormalTok{: }\FloatTok{0.05}\NormalTok{, }\DecValTok{3}\NormalTok{: }\FloatTok{0.2}\NormalTok{, }\DecValTok{4}\NormalTok{: }\FloatTok{0.2}\NormalTok{, }\DecValTok{5}\NormalTok{: }\FloatTok{0.2}\NormalTok{, }\DecValTok{6}\NormalTok{: }\FloatTok{0.2}\NormalTok{, }\DecValTok{7}\NormalTok{: }\FloatTok{0.2}\NormalTok{\}}

\CommentTok{\# Échantillonnage différencié par classe}
\NormalTok{sampled\_data }\OperatorTok{=}\NormalTok{ pd.concat([}
\NormalTok{    data[data[}\StringTok{\textquotesingle{}Cover\_Type\textquotesingle{}}\NormalTok{] }\OperatorTok{==}\NormalTok{ cls].sample(frac}\OperatorTok{=}\NormalTok{sampling\_rates[cls], random\_state}\OperatorTok{=}\DecValTok{42}\NormalTok{)}
    \ControlFlowTok{for}\NormalTok{ cls }\KeywordTok{in}\NormalTok{ data[}\StringTok{\textquotesingle{}Cover\_Type\textquotesingle{}}\NormalTok{].unique()}
\NormalTok{])}

\CommentTok{\# Réinitialisation des index après échantillonnage}
\NormalTok{sampled\_data }\OperatorTok{=}\NormalTok{ sampled\_data.reset\_index(drop}\OperatorTok{=}\VariableTok{True}\NormalTok{)}

\CommentTok{\# Affichage des effectifs par classe après échantillonnage}
\BuiltInTok{print}\NormalTok{(}\StringTok{"Effectifs par classe après échantillonnage différencié :"}\NormalTok{)}
\BuiltInTok{print}\NormalTok{(sampled\_data[}\StringTok{\textquotesingle{}Cover\_Type\textquotesingle{}}\NormalTok{].value\_counts().sort\_index())}

\CommentTok{\# Division des données en ensembles d\textquotesingle{}entraînement, validation et test}
\NormalTok{train\_data, temp\_data }\OperatorTok{=}\NormalTok{ train\_test\_split(sampled\_data, test\_size}\OperatorTok{=}\FloatTok{0.4}\NormalTok{, random\_state}\OperatorTok{=}\DecValTok{42}\NormalTok{, stratify}\OperatorTok{=}\NormalTok{sampled\_data[}\StringTok{\textquotesingle{}Cover\_Type\textquotesingle{}}\NormalTok{])}
\NormalTok{val\_data, test\_data }\OperatorTok{=}\NormalTok{ train\_test\_split(temp\_data, test\_size}\OperatorTok{=}\FloatTok{0.5}\NormalTok{, random\_state}\OperatorTok{=}\DecValTok{42}\NormalTok{, stratify}\OperatorTok{=}\NormalTok{temp\_data[}\StringTok{\textquotesingle{}Cover\_Type\textquotesingle{}}\NormalTok{])}

\CommentTok{\# Affichage des tailles des ensembles}
\BuiltInTok{print}\NormalTok{(}\SpecialStringTok{f"}\CharTok{\textbackslash{}n}\SpecialStringTok{Taille des ensembles :"}\NormalTok{)}
\BuiltInTok{print}\NormalTok{(}\SpecialStringTok{f"  {-} Entraînement : }\SpecialCharTok{\{}\BuiltInTok{len}\NormalTok{(train\_data)}\SpecialCharTok{\}}\SpecialStringTok{ lignes"}\NormalTok{)}
\BuiltInTok{print}\NormalTok{(}\SpecialStringTok{f"  {-} Validation : }\SpecialCharTok{\{}\BuiltInTok{len}\NormalTok{(val\_data)}\SpecialCharTok{\}}\SpecialStringTok{ lignes"}\NormalTok{)}
\BuiltInTok{print}\NormalTok{(}\SpecialStringTok{f"  {-} Test : }\SpecialCharTok{\{}\BuiltInTok{len}\NormalTok{(test\_data)}\SpecialCharTok{\}}\SpecialStringTok{ lignes"}\NormalTok{)}

\CommentTok{\# Affichage des effectifs par classe dans chaque ensemble}
\BuiltInTok{print}\NormalTok{(}\StringTok{"}\CharTok{\textbackslash{}n}\StringTok{Effectifs par classe dans l\textquotesingle{}ensemble d\textquotesingle{}entraînement :"}\NormalTok{)}
\BuiltInTok{print}\NormalTok{(train\_data[}\StringTok{\textquotesingle{}Cover\_Type\textquotesingle{}}\NormalTok{].value\_counts().sort\_index())}

\BuiltInTok{print}\NormalTok{(}\StringTok{"}\CharTok{\textbackslash{}n}\StringTok{Effectifs par classe dans l\textquotesingle{}ensemble de validation :"}\NormalTok{)}
\BuiltInTok{print}\NormalTok{(val\_data[}\StringTok{\textquotesingle{}Cover\_Type\textquotesingle{}}\NormalTok{].value\_counts().sort\_index())}

\BuiltInTok{print}\NormalTok{(}\StringTok{"}\CharTok{\textbackslash{}n}\StringTok{Effectifs par classe dans l\textquotesingle{}ensemble de test :"}\NormalTok{)}
\BuiltInTok{print}\NormalTok{(test\_data[}\StringTok{\textquotesingle{}Cover\_Type\textquotesingle{}}\NormalTok{].value\_counts().sort\_index())}

\CommentTok{\# Sauvegarder les ensembles en fichiers CSV}
\NormalTok{train\_data.to\_csv(}\StringTok{\textquotesingle{}covertype\_train.csv\textquotesingle{}}\NormalTok{, index}\OperatorTok{=}\VariableTok{False}\NormalTok{)}
\NormalTok{val\_data.to\_csv(}\StringTok{\textquotesingle{}covertype\_val.csv\textquotesingle{}}\NormalTok{, index}\OperatorTok{=}\VariableTok{False}\NormalTok{)}
\NormalTok{test\_data.to\_csv(}\StringTok{\textquotesingle{}covertype\_test.csv\textquotesingle{}}\NormalTok{, index}\OperatorTok{=}\VariableTok{False}\NormalTok{)}
\end{Highlighting}
\end{Shaded}

\begin{verbatim}
Effectifs par classe après échantillonnage différencié :
Cover_Type
1    10592
2    14165
3     7151
4      549
5     1899
6     3473
7     4102
Name: count, dtype: int64

Taille des ensembles :
  - Entraînement : 25158 lignes
  - Validation : 8386 lignes
  - Test : 8387 lignes

Effectifs par classe dans l'ensemble d'entraînement :
Cover_Type
1    6355
2    8499
3    4291
4     329
5    1139
6    2084
7    2461
Name: count, dtype: int64

Effectifs par classe dans l'ensemble de validation :
Cover_Type
1    2118
2    2833
3    1430
4     110
5     380
6     695
7     820
Name: count, dtype: int64

Effectifs par classe dans l'ensemble de test :
Cover_Type
1    2119
2    2833
3    1430
4     110
5     380
6     694
7     821
Name: count, dtype: int64
\end{verbatim}




\end{document}
