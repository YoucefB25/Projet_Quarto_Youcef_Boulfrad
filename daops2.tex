% Options for packages loaded elsewhere
\PassOptionsToPackage{unicode}{hyperref}
\PassOptionsToPackage{hyphens}{url}
\PassOptionsToPackage{dvipsnames,svgnames,x11names}{xcolor}
%
\documentclass[
  letterpaper,
  DIV=11,
  numbers=noendperiod]{scrartcl}

\usepackage{amsmath,amssymb}
\usepackage{iftex}
\ifPDFTeX
  \usepackage[T1]{fontenc}
  \usepackage[utf8]{inputenc}
  \usepackage{textcomp} % provide euro and other symbols
\else % if luatex or xetex
  \usepackage{unicode-math}
  \defaultfontfeatures{Scale=MatchLowercase}
  \defaultfontfeatures[\rmfamily]{Ligatures=TeX,Scale=1}
\fi
\usepackage{lmodern}
\ifPDFTeX\else  
    % xetex/luatex font selection
\fi
% Use upquote if available, for straight quotes in verbatim environments
\IfFileExists{upquote.sty}{\usepackage{upquote}}{}
\IfFileExists{microtype.sty}{% use microtype if available
  \usepackage[]{microtype}
  \UseMicrotypeSet[protrusion]{basicmath} % disable protrusion for tt fonts
}{}
\makeatletter
\@ifundefined{KOMAClassName}{% if non-KOMA class
  \IfFileExists{parskip.sty}{%
    \usepackage{parskip}
  }{% else
    \setlength{\parindent}{0pt}
    \setlength{\parskip}{6pt plus 2pt minus 1pt}}
}{% if KOMA class
  \KOMAoptions{parskip=half}}
\makeatother
\usepackage{xcolor}
\setlength{\emergencystretch}{3em} % prevent overfull lines
\setcounter{secnumdepth}{-\maxdimen} % remove section numbering
% Make \paragraph and \subparagraph free-standing
\makeatletter
\ifx\paragraph\undefined\else
  \let\oldparagraph\paragraph
  \renewcommand{\paragraph}{
    \@ifstar
      \xxxParagraphStar
      \xxxParagraphNoStar
  }
  \newcommand{\xxxParagraphStar}[1]{\oldparagraph*{#1}\mbox{}}
  \newcommand{\xxxParagraphNoStar}[1]{\oldparagraph{#1}\mbox{}}
\fi
\ifx\subparagraph\undefined\else
  \let\oldsubparagraph\subparagraph
  \renewcommand{\subparagraph}{
    \@ifstar
      \xxxSubParagraphStar
      \xxxSubParagraphNoStar
  }
  \newcommand{\xxxSubParagraphStar}[1]{\oldsubparagraph*{#1}\mbox{}}
  \newcommand{\xxxSubParagraphNoStar}[1]{\oldsubparagraph{#1}\mbox{}}
\fi
\makeatother


\providecommand{\tightlist}{%
  \setlength{\itemsep}{0pt}\setlength{\parskip}{0pt}}\usepackage{longtable,booktabs,array}
\usepackage{calc} % for calculating minipage widths
% Correct order of tables after \paragraph or \subparagraph
\usepackage{etoolbox}
\makeatletter
\patchcmd\longtable{\par}{\if@noskipsec\mbox{}\fi\par}{}{}
\makeatother
% Allow footnotes in longtable head/foot
\IfFileExists{footnotehyper.sty}{\usepackage{footnotehyper}}{\usepackage{footnote}}
\makesavenoteenv{longtable}
\usepackage{graphicx}
\makeatletter
\newsavebox\pandoc@box
\newcommand*\pandocbounded[1]{% scales image to fit in text height/width
  \sbox\pandoc@box{#1}%
  \Gscale@div\@tempa{\textheight}{\dimexpr\ht\pandoc@box+\dp\pandoc@box\relax}%
  \Gscale@div\@tempb{\linewidth}{\wd\pandoc@box}%
  \ifdim\@tempb\p@<\@tempa\p@\let\@tempa\@tempb\fi% select the smaller of both
  \ifdim\@tempa\p@<\p@\scalebox{\@tempa}{\usebox\pandoc@box}%
  \else\usebox{\pandoc@box}%
  \fi%
}
% Set default figure placement to htbp
\def\fps@figure{htbp}
\makeatother

\usepackage{fvextra}
\DefineVerbatimEnvironment{Highlighting}{Verbatim}{breaklines=true,breakanywhere=true,commandchars=\\\{\}}
\KOMAoption{captions}{tableheading}
\makeatletter
\@ifpackageloaded{caption}{}{\usepackage{caption}}
\AtBeginDocument{%
\ifdefined\contentsname
  \renewcommand*\contentsname{Table of contents}
\else
  \newcommand\contentsname{Table of contents}
\fi
\ifdefined\listfigurename
  \renewcommand*\listfigurename{List of Figures}
\else
  \newcommand\listfigurename{List of Figures}
\fi
\ifdefined\listtablename
  \renewcommand*\listtablename{List of Tables}
\else
  \newcommand\listtablename{List of Tables}
\fi
\ifdefined\figurename
  \renewcommand*\figurename{Figure}
\else
  \newcommand\figurename{Figure}
\fi
\ifdefined\tablename
  \renewcommand*\tablename{Table}
\else
  \newcommand\tablename{Table}
\fi
}
\@ifpackageloaded{float}{}{\usepackage{float}}
\floatstyle{ruled}
\@ifundefined{c@chapter}{\newfloat{codelisting}{h}{lop}}{\newfloat{codelisting}{h}{lop}[chapter]}
\floatname{codelisting}{Listing}
\newcommand*\listoflistings{\listof{codelisting}{List of Listings}}
\makeatother
\makeatletter
\makeatother
\makeatletter
\@ifpackageloaded{caption}{}{\usepackage{caption}}
\@ifpackageloaded{subcaption}{}{\usepackage{subcaption}}
\makeatother

\usepackage{bookmark}

\IfFileExists{xurl.sty}{\usepackage{xurl}}{} % add URL line breaks if available
\urlstyle{same} % disable monospaced font for URLs
\hypersetup{
  colorlinks=true,
  linkcolor={blue},
  filecolor={Maroon},
  citecolor={Blue},
  urlcolor={Blue},
  pdfcreator={LaTeX via pandoc}}


\author{}
\date{}

\begin{document}


Here are 100 key notions extracted and defined from the document:

\begin{center}\rule{0.5\linewidth}{0.5pt}\end{center}

\begin{enumerate}
\def\labelenumi{\arabic{enumi}.}
\item
  \textbf{DataOps}: A methodology for managing data pipelines that
  combines people, processes, and technology to deliver better results
  in a shorter time, emphasizing automation, collaboration, and
  continuous improvement.
\item
  \textbf{Data Lifecycle}: The sequence of stages data goes through from
  creation to archival or deletion, including collection, processing,
  analysis, and storage.
\item
  \textbf{Big Data}: Large volumes of data characterized by high
  velocity, variety, and volume, requiring advanced tools and techniques
  for processing and analysis.
\item
  \textbf{Data Pipeline}: A series of steps for processing data, where
  data moves from source to destination, undergoing transformations and
  analysis.
\item
  \textbf{DevOps}: A set of practices that combine software development
  (Dev) and IT operations (Ops) to shorten the development lifecycle and
  deliver high-quality software continuously.
\item
  \textbf{Agile Methodology}: A project management approach that
  emphasizes iterative development, collaboration, and flexibility to
  adapt to changing requirements.
\item
  \textbf{Lean Manufacturing}: A production methodology derived from
  Toyota's operational model, focusing on reducing waste and improving
  quality.
\item
  \textbf{Statistical Process Control (SPC)}: A method used in
  manufacturing to monitor and control process quality, applied in
  DataOps to ensure data quality.
\item
  \textbf{Data Governance}: A system for managing data assets, ensuring
  data quality, security, and compliance with organizational policies
  and regulations.
\item
  \textbf{Data Provenance}: The history of data, including its origin,
  transformations, and movements, used for tracking and quality
  assurance.
\item
  \textbf{Data Lineage}: The tracking of data from its source to its
  destination, including all transformations and processes it undergoes.
\item
  \textbf{ETL (Extract, Transform, Load)}: A process in data integration
  where data is extracted from sources, transformed into a suitable
  format, and loaded into a target system.
\item
  \textbf{ELT (Extract, Load, Transform)}: A variation of ETL where data
  is first loaded into a target system and then transformed.
\item
  \textbf{Continuous Integration (CI)}: A practice in software
  development where code changes are automatically tested and integrated
  into a shared repository.
\item
  \textbf{Continuous Deployment (CD)}: A practice where code changes are
  automatically deployed to production after passing tests.
\item
  \textbf{Apache Airflow}: An open-source workflow orchestration tool
  used to programmatically create, schedule, and monitor workflows.
\item
  \textbf{Docker}: A platform for developing, shipping, and running
  applications in containers, ensuring consistency across environments.
\item
  \textbf{Containerization}: The process of packaging an application and
  its dependencies into a container, allowing it to run consistently
  across different environments.
\item
  \textbf{Kubernetes}: An open-source platform for automating the
  deployment, scaling, and management of containerized applications.
\item
  \textbf{Data Versioning}: The practice of tracking and managing
  different versions of data, code, and artifacts to ensure
  reproducibility and traceability.
\item
  \textbf{Data Catalog}: A centralized repository that provides metadata
  about data assets, making it easier to discover, understand, and use
  data.
\item
  \textbf{Data Quality}: The measure of data's accuracy, completeness,
  consistency, and reliability, crucial for effective data analysis.
\item
  \textbf{Data Masking}: A technique used to protect sensitive data by
  obscuring it, ensuring privacy and compliance with regulations.
\item
  \textbf{Data Archiving}: The process of moving data that is no longer
  actively used to a separate storage system for long-term retention.
\item
  \textbf{Data Security}: Measures taken to protect data from
  unauthorized access, corruption, or theft, ensuring its
  confidentiality, integrity, and availability.
\item
  \textbf{Data Privacy}: The protection of personal data, ensuring that
  it is collected, processed, and stored in compliance with privacy laws
  and regulations.
\item
  \textbf{Data Analytics}: The process of examining data to extract
  insights, identify patterns, and support decision-making.
\item
  \textbf{Business Intelligence (BI)}: Technologies and practices for
  analyzing business data to provide actionable insights and support
  decision-making.
\item
  \textbf{Machine Learning}: A subset of artificial intelligence that
  involves training algorithms to make predictions or decisions based on
  data.
\item
  \textbf{Data Visualization}: The graphical representation of data to
  help users understand patterns, trends, and insights.
\item
  \textbf{Data Integration}: The process of combining data from
  different sources into a unified view, enabling comprehensive
  analysis.
\item
  \textbf{Data Warehousing}: The process of collecting, storing, and
  managing data from various sources in a centralized repository for
  analysis.
\item
  \textbf{Data Lake}: A storage repository that holds a vast amount of
  raw data in its native format until it is needed for analysis.
\item
  \textbf{Data Mining}: The process of discovering patterns and
  relationships in large datasets using statistical and machine learning
  techniques.
\item
  \textbf{Data Wrangling}: The process of cleaning, transforming, and
  organizing raw data into a usable format for analysis.
\item
  \textbf{Data Cleansing}: The process of detecting and correcting (or
  removing) corrupt or inaccurate data from a dataset.
\item
  \textbf{Data Transformation}: The process of converting data from one
  format or structure into another, often as part of the ETL process.
\item
  \textbf{Data Ingestion}: The process of collecting and importing data
  from various sources into a system for storage and analysis.
\item
  \textbf{Data Storage}: The retention of data in a structured or
  unstructured format, either temporarily or permanently, for future
  use.
\item
  \textbf{Data Archiving}: The process of moving data that is no longer
  actively used to a separate storage system for long-term retention.
\item
  \textbf{Data Backup}: The process of creating copies of data to
  protect against data loss due to hardware failure, corruption, or
  disasters.
\item
  \textbf{Data Recovery}: The process of restoring data from backups
  after data loss or corruption.
\item
  \textbf{Data Replication}: The process of copying data from one
  location to another to ensure availability and redundancy.
\item
  \textbf{Data Synchronization}: The process of ensuring that data in
  multiple locations is consistent and up-to-date.
\item
  \textbf{Data Migration}: The process of moving data from one system or
  storage location to another, often during system upgrades or
  consolidations.
\item
  \textbf{Data Compression}: The process of reducing the size of data to
  save storage space and improve transmission efficiency.
\item
  \textbf{Data Encryption}: The process of converting data into a coded
  format to prevent unauthorized access and ensure data security.
\item
  \textbf{Data Decryption}: The process of converting encrypted data
  back into its original format for authorized access.
\item
  \textbf{Data Partitioning}: The process of dividing a dataset into
  smaller, more manageable parts, often for performance optimization.
\item
  \textbf{Data Indexing}: The process of creating indexes for data to
  improve the speed of data retrieval operations.
\item
  \textbf{Data Querying}: The process of retrieving data from a database
  or data warehouse using query languages like SQL.
\item
  \textbf{Data Aggregation}: The process of collecting and summarizing
  data from multiple sources to provide a high-level view.
\item
  \textbf{Data Sampling}: The process of selecting a subset of data from
  a larger dataset for analysis, often to reduce processing time.
\item
  \textbf{Data Normalization}: The process of organizing data in a
  database to reduce redundancy and improve data integrity.
\item
  \textbf{Data Denormalization}: The process of combining data from
  multiple tables into a single table to improve query performance.
\item
  \textbf{Data Modeling}: The process of creating a data model that
  defines the structure, relationships, and constraints of data in a
  database.
\item
  \textbf{Data Schema}: The structure of a database, including tables,
  fields, and relationships, defined by a data model.
\item
  \textbf{Data Dictionary}: A document or repository that provides
  metadata about data, including definitions, formats, and
  relationships.
\item
  \textbf{Data Profiling}: The process of analyzing data to understand
  its structure, content, and quality.
\item
  \textbf{Data Validation}: The process of ensuring that data is
  accurate, complete, and consistent with predefined rules and
  standards.
\item
  \textbf{Data Reconciliation}: The process of comparing data from
  different sources to ensure consistency and accuracy.
\item
  \textbf{Data Auditing}: The process of reviewing and verifying data to
  ensure its accuracy, completeness, and compliance with regulations.
\item
  \textbf{Data Stewardship}: The management and oversight of data assets
  to ensure their quality, security, and proper use.
\item
  \textbf{Data Ownership}: The assignment of responsibility for managing
  and maintaining specific data assets within an organization.
\item
  \textbf{Data Access Control}: The process of restricting access to
  data based on user roles and permissions to ensure data security.
\item
  \textbf{Data Retention Policy}: A policy that defines how long data
  should be retained and when it should be archived or deleted.
\item
  \textbf{Data Destruction}: The process of permanently deleting data to
  ensure it cannot be recovered or accessed.
\item
  \textbf{Data Compliance}: The adherence to laws, regulations, and
  standards related to data management, privacy, and security.
\item
  \textbf{Data Ethics}: The moral principles and guidelines governing
  the collection, use, and sharing of data.
\item
  \textbf{Data Sovereignty}: The concept that data is subject to the
  laws and governance structures of the country in which it is located.
\item
  \textbf{Data Monetization}: The process of generating revenue from
  data by selling, licensing, or using it to create new products and
  services.
\item
  \textbf{Data Democratization}: The process of making data accessible
  to all users within an organization, regardless of their technical
  expertise.
\item
  \textbf{Data Literacy}: The ability to read, understand, and work with
  data, including the skills to analyze and interpret data.
\item
  \textbf{Data Science}: An interdisciplinary field that uses scientific
  methods, algorithms, and systems to extract knowledge and insights
  from data.
\item
  \textbf{Data Engineering}: The practice of designing and building
  systems for collecting, storing, and analyzing data at scale.
\item
  \textbf{Data Architecture}: The design and structure of data systems,
  including databases, data warehouses, and data lakes.
\item
  \textbf{Data Mesh}: A decentralized approach to data architecture that
  treats data as a product and emphasizes domain-oriented ownership.
\item
  \textbf{Data Fabric}: An integrated layer of data and connecting
  processes that provides a unified view of data across an organization.
\item
  \textbf{Data Virtualization}: The process of creating a virtual layer
  that provides access to data from multiple sources without physically
  moving or copying it.
\item
  \textbf{Data Streaming}: The continuous flow of data from sources to
  destinations, often in real-time, for immediate processing and
  analysis.
\item
  \textbf{Batch Processing}: The processing of data in large batches at
  scheduled intervals, often used for non-time-sensitive tasks.
\item
  \textbf{Real-Time Processing}: The immediate processing of data as it
  is generated, enabling real-time insights and actions.
\item
  \textbf{Data Orchestration}: The coordination and management of data
  workflows, ensuring that data moves smoothly through various stages of
  processing.
\item
  \textbf{Data Workflow}: A sequence of data processing tasks that are
  executed in a specific order to achieve a desired outcome.
\item
  \textbf{Data Pipeline Automation}: The use of tools and technologies
  to automate the execution of data pipelines, reducing manual effort
  and errors.
\item
  \textbf{Data Pipeline Monitoring}: The process of tracking the
  performance and status of data pipelines to ensure they are running
  smoothly and efficiently.
\item
  \textbf{Data Pipeline Testing}: The process of verifying that data
  pipelines are functioning correctly and producing accurate results.
\item
  \textbf{Data Pipeline Optimization}: The process of improving the
  performance and efficiency of data pipelines, often by reducing
  processing time and resource usage.
\item
  \textbf{Data Pipeline Scalability}: The ability of a data pipeline to
  handle increasing volumes of data and processing demands without
  degradation in performance.
\item
  \textbf{Data Pipeline Resilience}: The ability of a data pipeline to
  recover from failures and continue operating without data loss or
  corruption.
\item
  \textbf{Data Pipeline Security}: The measures taken to protect data
  pipelines from unauthorized access, data breaches, and other security
  threats.
\item
  \textbf{Data Pipeline Governance}: The policies and procedures for
  managing data pipelines, ensuring they comply with organizational
  standards and regulations.
\item
  \textbf{Data Pipeline Documentation}: The process of creating and
  maintaining documentation that describes the design, operation, and
  maintenance of data pipelines.
\item
  \textbf{Data Pipeline Versioning}: The practice of tracking and
  managing different versions of data pipelines to ensure
  reproducibility and traceability.
\item
  \textbf{Data Pipeline Collaboration}: The process of enabling multiple
  teams and stakeholders to work together on the design, development,
  and operation of data pipelines.
\item
  \textbf{Data Pipeline Integration}: The process of connecting data
  pipelines with other systems and tools to enable seamless data flow
  and processing.
\item
  \textbf{Data Pipeline Deployment}: The process of releasing data
  pipelines into production, ensuring they are ready for use and meet
  performance requirements.
\item
  \textbf{Data Pipeline Maintenance}: The ongoing process of monitoring,
  updating, and optimizing data pipelines to ensure they continue to
  operate effectively.
\item
  \textbf{Data Pipeline Troubleshooting}: The process of identifying and
  resolving issues in data pipelines to restore normal operation.
\item
  \textbf{Data Pipeline Lifecycle}: The stages a data pipeline goes
  through from design and development to deployment, operation, and
  eventual decommissioning.
\end{enumerate}

These notions cover a wide range of concepts related to DataOps, data
management, and data analytics, providing a comprehensive overview of
the key ideas discussed in the document.

\begin{center}\rule{0.5\linewidth}{0.5pt}\end{center}




\end{document}
